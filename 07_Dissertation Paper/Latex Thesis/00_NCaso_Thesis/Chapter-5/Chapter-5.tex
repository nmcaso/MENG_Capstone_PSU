\chapter{Conclusions}
\label{chapter:UQ_airfoil}
\graphicspath{{Chapter-5/graphics/}}

\section{Contributions}

\begin{enumerate}	
	\item Demonstrated the effectiveness of acquiring measurements in an anechoic chamber, rather than a wind-tunnel, which had previously shown to be unsuccessful. 
	
	\item Measurements were performed for a vortex ring convecting past the edge of a non-compact plate. Sound source waveform, directivity, and the power law relation between sound power and characteristic source speed, on one hand, and sound power and impact distance, on the other, were estimated from the measurements.
	
	\item Comparison of predictions of these quantities showed:
	\begin{enumerate}
		 \item Good agreement for the sound power-ring speed power law
		 \item Good agreement for the sound power-impact distance power law.
		 \item Qualitative agreement between acoustic directivity - while measurement showed a skewed-lobe dipole pattern, where the skewing decreased with porosity, the degree of skewing was over-predicted. 
		 \item Qualitative agreement in shape of source waveforms -- similar to directivity, predicted trends were observed but the degree of change with porosity was over-predicted.
	\end{enumerate}
\end{enumerate}